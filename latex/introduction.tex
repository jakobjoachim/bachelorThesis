Debugging software systems is a difficult task, to begin with. Debugging a distributed application makes this process even harder. Since the beginning of computer science, developers have always seen debugging as an unfortunate and tedious process. Always trying to minimise the time spent doing it by developing better programming styles and techniques. Unfortunately, not even the best programmer using the best possible method for his current project can write bug-free code all the time. A lot of programmers only learn how to debug by doing it; almost no one reads a book or scientific paper about how to improve ones debug technique. It makes sense to learn how to properly debug as a better debugging understanding leads to less time spent debugging and more time developing new features. As well as improving productivity, good knowledge of debugging also increases one's awareness of potential issues while writing code. So learning how to debug correctly not only reduces the time spent debugging but also enhances the software quality written by the developer.

\paragraph{} This thesis helps Flink developers to better understand how to debug Flink applications in a way that is less stressful and agonising then it would typically be, by following a simple but effective methodology.
