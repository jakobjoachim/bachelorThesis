Debugging software systems is a difficult task, to begin with. Debugging a distributed application makes this process even harder. Since the beginning of computer science, developers have always seen debugging as an unfortunate and tedious process. Always trying to minimise the time spent doing it by developing better programming styles and techniques. Unfortunately, not even the best programmer using the best possible method for his current project can write bug-free code all the time. A lot of programmers only learn how to debug by doing it; almost no one reads a book or scientific paper about how to improve one's debug technique. Barry W. Boehm estimates that reworking defects in requirements, design, and code consumes 40-50\% of the total cost of software development \cite{1663694}. It makes sense to learn how to properly debug as a better debugging understanding leads to less time spent debugging and more time developing new features. As well as improving productivity, good knowledge of debugging also increases one's awareness of potential issues while writing code. So learning how to debug correctly not only reduces the time spent debugging but also enhances the software quality written by the developer.

\section{Objective of the Thesis}
Flink is a stream processing framework designed to make it easier for developers to write distributed applications that have an continuous input (stream) of data. Even though applications for Flink are much easier to understand, write and debug it is still far from easy. This thesis tries to make it simpler for Flink developers to find bugs in their code. This is done by providing a methodology for debugging Flink and a debugging tool to the developer as well as some general recommendations for building Flink applications. When finished with this thesis the developer should have a good understanding as to why an error might occur even if the error message itself is not helpful.

\pagebreak

\section{Structure of the Thesis}
In \emph{chapter two}, debugging techniques are discussed and a methodology for debugging applications of all kinds is outlined\\
In \emph{chapter three}, the Flink framework is explained, and similar work is analysed.\\
In \emph{chapter four}, a debugging methodology for Flink is explained and the relevant steps shown\\
In \emph{chapter five}, the Flink Backtracker tool is shown, and its internals discussed.\\
The final \emph{chapter six} sums up the thesis, explains what lessons were learned and future work for debugging Flink applications is presented.
