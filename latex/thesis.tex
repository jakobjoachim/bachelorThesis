\documentclass[draft=false
              ,paper=a4
              ,twoside=false
              ,fontsize=11pt
              ,headsepline
              ,BCOR10mm
              ,DIV11
              ]{scrbook}
\usepackage[ngerman,english]{babel}
%% see http://www.tex.ac.uk/cgi-bin/texfaq2html?label=uselmfonts
\usepackage[T1]{fontenc}
\usepackage[utf8]{inputenc}
\usepackage{libertine}
\usepackage{pifont}
\usepackage{microtype}
\usepackage{textcomp}
\usepackage[german,refpage]{nomencl}
\usepackage{setspace}
\usepackage{makeidx}
\usepackage{listings}
\usepackage{natbib}
\usepackage[ngerman,colorlinks=true]{hyperref}
\usepackage{soul}
\usepackage[printer]{hawstyle}

%% define some colors
\colorlet{BackgroundColor}{gray!20}
\colorlet{KeywordColor}{blue}
\colorlet{CommentColor}{black!60}
%% for tables
\colorlet{HeadColor}{gray!60}
\colorlet{Color1}{blue!10}
\colorlet{Color2}{white}

%% configure colors
\HAWifprinter{
  \colorlet{BackgroundColor}{gray!20}
  \colorlet{KeywordColor}{black}
  \colorlet{CommentColor}{gray}
  % for tables
  \colorlet{HeadColor}{gray!60}
  \colorlet{Color1}{gray!40}
  \colorlet{Color2}{white}
}{}
\lstset{%
  numbers=left,
  numberstyle=\tiny,
  stepnumber=1,
  numbersep=5pt,
  basicstyle=\ttfamily\small,
  keywordstyle=\color{KeywordColor}\bfseries,
  identifierstyle=\color{black},
  commentstyle=\color{CommentColor},
  backgroundcolor=\color{BackgroundColor},
  captionpos=b,
  fontadjust=true
}
\lstset{escapeinside={(*@}{@*)}, % used to enter latex code inside listings
        morekeywords={uint32_t, int32_t}
}
\ifpdfoutput{
  \hypersetup{bookmarksopen=false,bookmarksnumbered,linktocpage}
}{}

%% more fancy C++
\DeclareRobustCommand{\cxx}{C\raisebox{0.25ex}{{\scriptsize +\kern-0.25ex +}}}

\clubpenalty=10000
\widowpenalty=10000
\displaywidowpenalty=10000

% unknown hyphenations
\hyphenation{
}

%% recalculate text area
\typearea[current]{last}

\makeindex
\makenomenclature

\begin{document}
\selectlanguage{english}

%%%%%
%% customize (see readme.pdf for supported values)
\HAWThesisProperties{Author={Jakob Joachim}
                    ,Title={Methodology for Debugging Flink Applications}
                    ,EnglishTitle={Methodology for Debugging Flink Applications}
                    ,ThesisType={Bachelorarbeit}
                    ,ExaminationType={Bachelorprüfung}
                    ,DegreeProgramme={Bachelor of Science Angewandte Informatik}
                    ,ThesisExperts={Prof. Dr. Olaf Zukunft \and Prof. Dr. Klaus-Peter Kossakowski}
                    ,ReleaseDate={2. February 2018}
                    }
%% title
\frontmatter

%% output title page
\maketitle

\onehalfspacing

%% add abstract pages
%% note: this is one command on multiple lines
\HAWAbstractPage
%% German abstract
{Flink, Debugging, Stream Processing, Methodik zum Debuggen, Java Debuggen, Flink Debuggen}%
{Das Debuggen von Programmen ist kompliziert, wird aber in einem verteilten System noch komplizierter. Je größer die täglich erzeugten Datenmengen werden, desto wichtiger wird die Nutzung verteilter Anwendung. Flink ist ein Framework zur verteilten Verarbeitung von Daten aus einem Stream, wobei die Daten zwischen den einzelnen Verarbeitungsschritten ebenfalls durch Streams transportiert werden. Flink erleichtert es dem Entwickler, verteilte Anwendungen, die auf der Datenverarbeitung basieren, zu entwickeln. Diese Thesis stellt eine Methodik zum Debuggen von eben diesen Flink Anwendungen dar.}
%% English abstract
{Flink, Debugging, Stream Processing, Methodology for Debugging, Debugging Java, Debugging Flink}%
{Debugging Applications is hard, debugging distributed applications is harder and the more massive the amount of data that gets generated every day gets, the higher the need for these distributed applications. Flink is a distributed stream processing framework; it processes data while only using streams. It makes it easy for developers to write these distributed applications to handle the ever-growing amounts of data in the world. This thesis explains how to debug these Flink applications by providing a debugging methodology.}

\newpage
\singlespacing

\tableofcontents
\newpage
%% enable if these lists should be shown on their own page
%%\listoftables
%%\listoffigures
\lstlistoflistings
\listoffigures

%% main
\mainmatter
\onehalfspacing
%% write to the log/stdout
\typeout{===== File: chapter 1}
\chapter{Introduction}
  Debugging software systems is a difficult task, to begin with. Debugging a distributed application makes this process even harder. Since the beginning of computer science, developers have always seen debugging as an unfortunate and tedious process. Always trying to minimise the time spent doing it by developing better programming styles and techniques. Unfortunately, not even the best programmer using the best possible method for his current project can write bug-free code all the time. A lot of programmers only learn how to debug by doing it; almost no one reads a book or scientific paper about how to improve one's debug technique. Barry W. Boehm estimates that reworking defects in requirements, design, and code consumes 40-50\% of the total cost of software development \cite{1663694}. It makes sense to learn how to properly debug as a better debugging understanding leads to less time spent debugging and more time developing new features. As well as improving productivity, good knowledge of debugging also increases one's awareness of potential issues while writing code. So learning how to debug correctly not only reduces the time spent debugging but also enhances the software quality written by the developer.

\section{Objective of the Thesis}
Flink is a stream processing framework designed to make it easier for developers to write distributed applications that have an continuous input (stream) of data. Even though applications for Flink are much easier to understand, write and debug it is still far from easy. This thesis tries to make it simpler for Flink developers to find bugs in their code. This is done by providing a methodology for debugging Flink and a debugging tool to the developer as well as some general recommendations for building Flink applications. When finished with this thesis the developer should have a good understanding as to why an error might occur even if the error message itself is not helpful.

\pagebreak

\section{Structure of the Thesis}
In \emph{chapter two}, debugging techniques are discussed and a methodology for debugging applications of all kinds is outlined\\
In \emph{chapter three}, the Flink framework is explained, and similar work is analysed.\\
In \emph{chapter four}, a debugging methodology for Flink is explained and the relevant steps shown\\
In \emph{chapter five}, the Flink Backtracker tool is shown, and its internals discussed.\\
The final \emph{chapter six} sums up the thesis, explains what lessons were learned and future work for debugging Flink applications is presented.


\typeout{===== File: chapter 2}
\chapter{The Art of Debugging}
  As debugging of traditional programs is in many ways similar to debugging Flink applications it is necessary to explore what good debugging is. This chapter is an accumulation of techniques and methods that many people consider essential to debug effectively.

\section{Word Count Application}
\label{wordCountApplication}
This section quickly outlines the "Word Count Application" that is used in the thesis as an example. The Application counts how often each word is in a spcific text and returns the results to the command line. An example run of the program would be:

\begin{lstlisting}
  $ java -jar wordCountSimple.jar "Hello hello how do you do"
  how: 1
  hello: 2
  do: 2
  you: 1
\end{lstlisting}

Note that the application ignores capital letters in the input. This will be relevant later on.

\section{Why Programs Fail}
This section will examine the "TRAFFIC" method of debugging that is published in \cite{Zeller:2009:WPF:1718010}'s book "Why Programs Fail". It consists of seven steps that lead to a quick removal of the failure while still preserving the information on how the failure came to be as a big part of debugging is not to fix a failure but to make sure that the same or a similar failure won't happen again.
The seven steps are:
\begin{enumerate}
  \item Track the problem in the Database
  \item Reproduce the failure
  \item Automate and simplify the test case
  \item Find possbile infection origins
  \item Focus on the most likely origins
  \item Isolate the infection in the chain
  \item Correct the defect
\end{enumerate}

The following subsection explaines each step of the method along the example program outlines above \ref{wordCountApplication}

\subsection{Track the problem in the Database}
\label{aodZellerTrack}
Tracking is not the first step, but a good method at each step, logging what happened so that anybody involved knows how far each problem was investigated. It is started once someone finds a problem. Put simply tracking is holding the information on what the problem is and how close the developer is to fix it. This is mostly done on a platform that is accessible for all involved parties to make the communication between the user and the developer easier. This platform is only useful when used permanently as out of date information is more harmful than useful. Most software projects have multiple people working on them, not all of them know every bit of the program. This adds complexity to an already complicated process. It is necessary to split the work to different developers andd this process has to be logged otherwise problem reports might get lost. A developer might have 20 problem reports on his desk of which he only knows how to solve a few. Other reported problems might not even be problems, but a wanted state. For example, someone might report that a password field is showing only stars instead of the letters he put in. As this (for the developer) is a security feature, it will not be changed and has to be logged or noted so that the same problem will hopefully not be reported again. To solve these and other issues tracking should be used. Depending on the application these metrics are useful:
\begin{enumerate}
  \item The State of the Problem - Is the problem new, assigned to a developer,
resolved, closed, etc. This is useful for the developer as he can easily see which problems he has to work on and which are already solved. It is also beneficial for the user as he can easily see when his problem is resolved.
  \item The Resolution - Is the problem fixed, invalid, won't be fixed (as the example above), a duplicate, etc. This is useful as it lets the user see to which conclusion the developer came.
  \item Assigned Developer -  Which developer is assigned to the problem. Makes it easy to communicate with the correct person and lets people know that the problem is worked on.
  \item Severity - Is the problem crucial or is it only a minor inconvenience. Helps the developer prioritise which problems to solve first.
\end{enumerate}

These are just the most important once, depending on the project more should be added.

\subsection{Reproduce the failure}
\label{aodZellerReproduce}
The first real step in any debugging activity is to consistently reproduce the problem described in the problem report. This has two important reasons:

\begin{enumerate}
  \item To observe the problem - The developer has to be able to reproduce the problem to fix it, the developer could also check the source code at a position he thinks could be responsible for the problem without reproducing it, but that makes it unnecessarily hard on the developer as a good problem report should be re-creatable.
  \item To check whether the problem is fixed - It is incredibly hard to tell in most situations if a problem is solved or not without being able to rerun the problem without it happening.
\end{enumerate}

Reproducing a problem can be incredibly hard, as the problem is rarely found by the person that has to fix it, but by a person who doesn't understand how the program works. This makes it difficult for both the finder and the fixer of the problem, as the finder doesn't know which information the fixer might need, and for the fixer as he can only reproduce it with the corresponding information.

Reproducing is done by going through the following three steps until the problem can be reproduced:
\begin{enumerate}
  \item Reproduce the Problem locally - In the best case scenario, the problem can be replicated on the local machine of the developer fixing it with the information provided by the problem report. This is most successful when the problem state is not connected with many other choices in the program. For example, a button might not work (as reported in the problem report) that starts some function no matter what else happened before the button was pressed.
  \item Adopt more Circumstances of the Problem environment - Sometimes the problem can not be reproduced by only following the steps provided by the problem report. In that case, it is necessary to check what else is known about the environment the problem was found in. This means installing the same version of depending software, using the same configuration file, using the same hardware or anything else that could influence the described problem.
  \item Contact the Problem Finder or declare the Problem invalid - If the reproduction failed even with all circumstances applied as outlined in the problem report, there are only two options left. Either the problem is not real, or the finder has not provided the necessary information. Depending on the bug this can be solved by contacting the finder and asking him for more details. It is beneficial for the developer to see the problem live, so it would be a good option to remotely gain access to the computer of the finder or get a video meeting with him. This is obviously not possible or feasible in many situations but extremely helpful if it can be arranged.
\end{enumerate}

At this point, it is important to mark what information were necessary, so that another developer can reproduce the problem easier as well as saving the information as it is not guaranteed that the problem will be solved immediately. This is done with tracking: see \ref{aodZellerTrack}

\subsection{Automate and simplify the test case}
Once the problem is reproduced, it is desired to simplify the problem so that it can be replicated by a test case that as well as confirming that the problem is solved, helps avoid building a similar problem at a later time.

Simplification is done by understanding what the root of the problem is. When the problem occurs when pressing a particular button, but the button has no context to other set states of the program it is safe to assume that actions done before the button click can be omitted. Thus leaving only the test case: After button press, function x started? This statement can now be easily modelled by a test case.

The simplification is not only essential for building test cases but also helpful for the developer when trying to locate the problem in the code as it is much easier to find the error when only looking at the relevant pieces of source code.

\paragraph{Why write tests?}
It is understandable to ask why to write a test when it is already known where the bug is originating from. The developer only has to fix the problem and manually test the program once, what good is a test case here?
It is important to understand that finding a problem is giving a lot of valuable information to the developer he might not even realise. We can assume that if the problem got into the code once it might get into the code again. The only way to make sure it does not, is to give the developer an imidiate feedback when the problem occurs again. As well as helping the Developer in the future it can assist the developer while writing the fix to quickly test if the fix worked or not without having to manually start the application and reproduce the problem all over again.

\lstinputlisting[language=Java]{WordCountBug.java}

\subsection{Find possbile infection origins}
Once the failure can be reproduced easily, it is time to search for what part of the code is responsible for the failure. This is often the hardest part of debugging. To make finding easier for the developer the traffic approach suggests to use backtracking to find the relevant piece of code. Backtracking is done by starting at the manifestation of the error, meaning if the program fails with an exception, the line noted in the exception is used as a start point. If the program just produces an unwanted or wrong value, the line is used where that value was returned. Once a starting point is found the next step is to back track all active and passive usages of the variable. Lines that don't use the Variable can be omitted as they cannot be responsible for the failure. The information gained by this technique should be displayed by a control flow graph to make them easier to understand.


\subsection{Focus on the most likely origins}
\subsection{Isolate the infection in the chain}
\subsection{Correct the defect}


\typeout{===== File: chapter 3}
\chapter{Related Work}
  The goal of this section is to give the necessary information to understand the rest of the thesis. Debugging is a complicated process that is made much harder by distributing the system to more than one physical location.

\section{Flink Framework}
\label{flinkFramework}

Flink is a Framework for processing applications that are distributed across multiple computers. This part will lay out the foundations of the Flink Framework.
Flink applications have a simple base structure that is used by the developer. Each application defines tasks which have a particular structure. Each task has an input and output stream. These are called source and sink. Because both the source and sink of these tasks are streams, they can be attached to each other thus creating a data flow from task to task until the wanted end state is reached.

\begin{figure}[h!]
    \centering
      \includegraphics[width=0.9\textwidth]{rw_simpleDataflowInFlink.png}
      \caption{"Simple Dataflow in Flink"}
      \label{simpleDataflowInFlink}
\end{figure}

Figure \ref{simpleDataflowInFlink} shows a basic program flow. On the left side, it starts with a source. The source is then transferred over a data stream to the first task. The map operation is executed, and the result is sent over a data stream to the next task where the same procedure will start again. The resulting data in this example is the sink, meaning we reached the end of the program. The sink will typically be connected to a database or some other kind of Technology for preserving the data. The most basic option would be to write the sink onto the standard output on the console.

Until now the system can only be distributed by having the different tasks on different physical computers. This distribution will not suffice when the amount of input data gets too high. To achieve real distribution in the application, each task can be run multiple times as so called subtasks. These subtasks run as a single thread in the JVM and are managed by a task manager. Task Manager connect to a Job Manager which coordinates the distributed execution. The data flow from one subtask to another can either be one to one or a redistributing data flow. A redistributing data flow is necessary to achieve an even distribution of data at the next subtasks.

\begin{figure}[h!]
    \centering
      \includegraphics[width=0.9\textwidth]{rw_distributedDataflowInFlink.png}
      \caption{"Distributed Dataflow in Flink"}
      \label{distributedDataflowInFlink}
\end{figure}

Figure \ref{distributedDataflowInFlink} presents the same example as figure 1.1 just with the subtasks shown as well. The map task splits the data


\typeout{===== File: chapter 4}
\chapter{Debugging Flink Applications}
  
This chapter is the centre part of the thesis. It explains in detail how the method works. It consists of two main sections, the first of which explores in detail what a developer should do while writing the program to minimise the work when debugging. This step is crucial as it is much harder to recreate problems locally as opposed to regular Java applications. The second part focuses on the debugging itself once a developer is informed of an error and has to figure out what is causing it.

\section{Better Developing}
Flink applications are run remotely and without an active user providing input as it is typical for a regular application. Not having a user makes it much harder to recreate the problem as we only have the log files and the stack trace as information. As such it is crucial to have all the information at hand when the program fails, or we notice discrepancies in the resulting data. The only way to make sure that the information is accessible once a problem is reported is to think about what data is necessary for the debugging developer while writing the program. Another issue is that some problems are unique to a distributed environment and won't happen when running on a local host. This section will explain what can be done while writing the program to make the debugging progress much easier.

\subsection{Building Tasks}
The Flink Framework promises a few features that try to set it apart from standard Java applications. These are referenced in \ref{flinkFramework}. The one that sets these applications apart from standard java once while writing the application is the distribution. Applications should be built in a modular way to take full advantage of these features and also make the program easily debuggable.
The development of these tasks should be done using a very similar set of rules as the Unix philosophy states. In fact, Flink modules are not too different to Unix command line tools, they both provide a service while taking an input/output in a predefined way. For the Unix command line, this is the standard output, for Flink programs, these are input and output streams.

\subsubsection{Unix Philosophy}
The Unix philosophy is defined by Doug McIlroy, the inventor of the Unix pipe as follows: \cite{bell1978}
\begin{enumerate}
  \item Make each program do one thing well. To do a new job, build afresh rather than complicate old programs by adding new features.
  \item Expect the output of every program to become the input to another, as yet unknown, program. Don't clutter output with extraneous information. Avoid stringently columnar or binary input formats. Don't insist on interactive input.
  \item Design and build software, even operating systems, to be tried early, ideally within weeks. Don't hesitate to throw away the clumsy parts and rebuild them.
  \item Use tools in preference to unskilled help to lighten a programming task, even if you have to detour to build the tools and expect to throw some of them out after you've finished using them.
\end{enumerate}

Most of the points mentioned here are of some relevance for Flink programming as well. Each task should do one job and do it well. The next paragraph will look into why that is. The second point is necessary by default in Flink, each task has to use the provided streams to work. The third point is equally important if not more important in Flink applications as it is in Unix programs. Always test each Task individually to make sure it works as designed and has no flaws on its own, only then can the whole application work without any problems.

Dividing the program into various modules has a lot of advantages:
\begin{enumerate}
  \item Easier to Develop - It is much simpler to develop a smaller application as it is much harder to lose track of what each piece of code should do. This reduced complexity in return reduces the likelihood of mistakes. Once the program is completed, it results in a more stable program that can be debugged easier.
  \item Better Distribution -  As every Flink task can run on a different computer the smaller the tasks are, the better the Flink Job Manager can distribute the load evenly between the available resources.
  \item Checkpoints are easy to find - Checkpoints are a core piece of Flink technology. It allows the framework not only to jump back and repeat a failed run without having to restart the whole application but also provides information about which state the application is currently in. This is extremely helpful as a lot of Flink applications only end when the program is cancelled by the user.
\end{enumerate}

The modulation of the program not only helps to achieve the advantages of the framework but also supports with debugging later as a lot of the information needed are gathered at the checkpoints.

\subsubsection{Where to split the program}
It should now be understandable that the programs should be divided into multiple tasks, the next question now is how to break the program to get a simple program where there are enough tasks but not too many as too many would lead to the opposite effect we want to achieve.

\paragraph{Why are too many tasks bad?} When there are too many tasks, it gets even more complicated than when everything would be in its task as basically every line of code would be in a different place. Additionally, it wouldn't increase the performance as each task has some initialisation work that would diminish the performance gain achieved by distributing it perfectly.

It makes sense to use the Unix philosophy of having one task do one thing. In most cases there are some obvious choices as in most Flink programs data is modified or analysed each task could be one transformation of the data. It should also be stated that when ever possible the pre-existing transformation functions of the Flink framework should be used. Only in rare situations is it necessary to implement your own data transformation classes. The predefined classes have the advantage of being extensively tested in different conditions thus the risk of data going missing is extremely low.

\subsection{Metrics}
Now that the architecture of the program is done the next question is what metrics to use in which positions to achieve the optimal security.

Once the program architecture is finished, it makes sense to think about what kind of metrics can be used where. Metrics are used to monitor the program without having to debug it and are crucial in notifying the developer if something looks wrong. The first step is figuring out where to use metrics. A good start is to look at the application in the worst possible way; what is the most likely area that an error will occur, after that it makes sense to surround the area with metrics that will catch and log the gathered data. Another great location for metrics is at positions where the through coming data is simple, and metrics can easily be implemented. This should be the case in between tasks. As optimally each task only does one thing it should be easy to check whether the starting and ending assertions are valid.

\subsection{Logging}
Logging in Flink is straightforward and can is used the same as in every other java program that uses log4j. As Flink already provides the necessary libraries to use log4j all a developer has to do is to write the logging config file. Although the logging process itself is the same as every other Java application, it should not be forgotten to use the different log levels that log4j provides. There is a lot of logging happening out of the box just by the Flink process itself. The six logging levels, from highest to lowest are:
\begin{enumerate}
  \item FATAL - the highest logging level, should only be used when the application cannot continue to work because of an unexpected error.
  \item ERROR - whenever an unexpected exception is thrown it should be logged.
  \item WARN - warnings that could lead to errors later on. These are difficult to think of beforehand but if used correctly are very valuable for the debugging developer.
  \item INFO - relevant information like successful database connections and other milestones in the application to let the reader of a log file understand at which point in an application the program currently is.
  \item DEBUG - should be used to record relevant information along the way that could be useful to a programmer when debugging. This could, for example, contain values of variables like database connection strings.
  \item TRACE - is used to let a developer searching for a bug understand the path the application took. Should be logged into a unique log file as it would flood every other one.
\end{enumerate}

\section{Titel to do (Debugging Flink once an error occurs)}

There are multiple reasons why an error might occur, the most common one being an exception in a log file. Another option is that the end user of the results finds that some of the results are incorrect. Both of these cases require slightly different handling. An excellent way to start the debug process is by using a modified version of the Traffic approach introduced here: \ref{aodZeller}.

\subsection{Track the Problem in the database}
Tracking a problem is essential in every development cycle no matter in which language or with what framework and Flink is no exception. It is crucial for every developer to track the status of problems in Flink applications as it helps to minimise work. Along the already mentioned advantages in chapter \ref{aodZellerTrack}, like having an easily accessible database of open problems and knowing which problems are more important than others, tracking the problems of flink applications offers some other advantages as well.

\begin{enumerate}
  \item Having access to relevant log files.
  \item Knowing how past problems were solved.
  \item What relevant metrics were when the error occurred.
  \item Which subtask of what task manager failed, allows seeing if problems only occur on one machine.
\end{enumerate}

To achieve these advantages the tracking database needs additionally to the already mentioned fields in \ref{aodZellerTrack} a few additional columns. As soon as a problem is experienced the current log files should be saved so that it is easy for the developer to find the relevant lines in the log file even if he starts debugging months later. Secondly, for the same reason, all appropriate metrics should be saved as well.

The resulting columns now are:

\begin{enumerate}
  \item Description
  \item State
  \item Resolution
  \item Assigned Developer
  \item Severity
  \item Link to logs
  \item Steps that were taken to resolve the problem
  \item Relevant metrics
  \item Task manager that was used
\end{enumerate}

\subsection{Reproduce the failure}
Reproducing the problem is probably the most challenging part of debugging flink applications. As there is no user to report the problem the only help the debugging developer has are information provided by the problem report from the last chapter.

There are two options how a problem is discovered, and both require different steps to reproduce the problem. The first and more difficult one is that an error in the resulting data is discovered without an exception being recorded in the log files. This means that the program is doing something different then what the developer expected when writing it. The second option to discover a problem is by having an exception showing up in the log file.

\subsubsection{Faulty resulting data}
This section will use the word count application as an example program to debug. Notice that the exact implementation of it is irrelevant at this point. The application is a black box as only the incoming, and resulting data are known. To always have the same expected result the following sentence will be used as an input each time: "Hello hello flink flink flink one two". The expected result would be:
\begin{lstlisting}
  hello - 2, flink - 3, one - 1, two - 1
\end{lstlisting}

The first question that has to be answered is
"how much data is affected". Are only a few pieces incorrect or is everything faulty? Imagine the result of the application would be:
\begin{lstlisting}
  hello - 3, flink - 1, one - 1, two - 2
\end{lstlisting}
  Each word has the count of the next word. So this would be considered as the second case "everything is faulty" even though the word "one" has the correct answer. This is important to note as sometimes a fault in a program can still result in a correct result. This first question just differentiates between a few faults and a majority of faults, so the debugging developer has to look at the whole picture and see if the majority of data is corrupt. An example for only a few faults would be:
\begin{lstlisting}
 Hello - 1, hello - 1, flink - 3, one - 1, two - 1
\end{lstlisting}
Here only one additional word was counted ("Hello").
\paragraph{} If the first case is valid (all data is faulty) the next question that should be asked is why the problem is only now showing up. If the problem is observable for almost all the resulting data, surely it should have been noticed while testing the application. In most cases, the problem was either found during testing in which case the problem is already reproducible or was not observable on the local test machine. That means that either the incoming data is different to the local one or that something is being executed differently on the remote network than on the local machine. As Flink is responsible for the distribution and everything is running in a JVM, it is implausible that flink is to blame. In most cases, the data on the server will differ from the local one. If that can be confirmed the only thing left to do is to update the local test data so that the problem can be reproduced locally.

The other option was that only some pieces of data were wrong. In that case, the process of reproducing the fault is entirely different. There are two options available. First, figure out what makes the faulty data unique in comparison to the other data. In the word count example, this would be the capital "H" at the beginning of the first "Hello". If this option is successful, the unique case can be added to the test cases, and the reproduction was successful. If on the other hand, the developer can't figure out why the one failing case is different to the others the tool that is written alongside this thesis can be used, it will be explained in detail later on. It can show which incoming data was leading to which result. In the example above it could show that the first "hello" was the result of the original sentence. As there is only one sentence in this example that is not very helpful, but in a more realistic use case, there could be millions of sentences where just a few have capital letters in them. Once the starting sentence is discovered, it can easily be reproduced.

\paragraph{} The fault should now be reproducible on a local machine as the affecting test data was found. The only fault that remains are problems that only occur on the remote network and are not happening because of incoming data.

\subsubsection{Exception in log file}


\typeout{===== File: chapter 5}
\chapter{Flink Backtracker}
\label{flinkBacktracker}
  This chapter will explain in detail how the Flink Backtracking tool written alongside this thesis works and how it is used. It is composed of two parts, the frontend part that shows the results in the IDE and the backend part that has to be included in the project that sends the information to the frontend. This chapter is thus split into multiple paths:
\begin{enumerate}
  \item[\ref{fbManual}] User manual
  \item[\ref{fbBackend}] Backend architecture
  \item[\ref{fbFrontend}] Frontend architecture
  \item[\ref{fbState}] State of the program
\end{enumerate}

\section{User manual}
\label{fbManual}

\paragraph{Introduction to the Flink Backtracking Tool}
The Flink Backtracking Tool allows the user to see what data was read at each data stream and what data lead to the result at the next data stream. The Flink Watchpoint mechanism is used which injects a barrier with an identifier ever so often. The data between each of these barriers at each data stream is the result of the data between the same barriers plus the transformation of the previous data stream. Thus the exact result of each data stream can be seen which is not possible with step by step debugging.

\paragraph{Prerequirements} There are a few requirements that have to be met to make the program work appropriately. First and foremost it has to be a Flink application as it can only track Flink data streams. Also because it is using the watchpoint mechanism of Flink all requirements that go along with that have to be met. This means each data stream has to support rolling back data in cases of failure. Because of this requirement, it is currently only possible to use data that comes from some big data storage. In the following examples, Apache Kafka will be used.

Using the program is an easy process. The "FlinkBacktracking.jar" has to be integrated into the Flink application that is being developed, and the "FlinkBacktrackingIntelliJPlugin.jar" has to be activated in IntelliJ to see the results of the Tool. There is only one line needed to enable the tool itself, although checkpointing still has to be enabled manually:

\begin{lstlisting}[caption={How to use the Backtracker}]
env.setStreamTimeCharacteristic(TimeCharacteristic.EventTime);
env.enableCheckpointing(1000);
FlinkBacktrack backtracker = new FlinkBacktrack(env);
\end{lstlisting}

Where "env" is the Execution Environment. Any number of watchpoint behaviour is possible, in this example it is set to inject a barrier after the specified amount of time. Once the Backtracker is initialised, it can be used by adding each data stream that the developer wants to be watched to the tool. This is done in the following way:

\begin{lstlisting}
backtracker.track(dataStream, "Transition name");
\end{lstlisting}

Where "dataStream" is the dataStream that should be tracked and "Transition name" is an arbitrary name that is used to show the results under. This step has to be repeated for each data stream that should be watched. Once these steps are done, the developer can open the "Flink Backtracking View" by clicking on the button with the same name on the lower right side of IntelliJ and start the application.

\subsection{Debugging a Program with the Backtracker}
Once the plugin has been configured the application can be started. Each piece of data that is being processed by the application will show up in the "Flink Backtracker" tab in IntelliJ as well. The first tabs that are shown are the different data streams. Each watched data stream has its own tab. Each of these tabs contains a list of watchpoints that passed since the program was started and each of these includes the data that was processed by that data stream at the time of the watchpoint.

\begin{figure}[h!]
    \centering
      \includegraphics[width=0.9\textwidth]{FlinkBacktrackerIntelliJPluginExample.png}
      \caption{Backtracking Example}
      \label{flinkBacktrackingExample}
\end{figure}

\paragraph{} The example \ref{flinkBacktrackingExample} shows a good use-case for the tool. The developer can easily see that the input of the program was successfully received by the first data stream. He can also see that the application is removing special characters and converting all capital letters. Because the content of the second data stream (result) is not only depending on the data stream that came before (input) but also on all the data that was processed by the stream previously. Thus the "hello,2" in the result transition at watchpoint 4316.

\section{Backend architecture}
\label{fbBackend}
The Backend is composed of two parts: One is the function that is collecting the data and sending it to the frontend the other is the main class that is used to set up the function in the correct way to ensure that the function can gather all the necessary information.

\paragraph{} The following example (\ref{dataFlowNoBacktracker}) helps to underline how the Flink application is set up so that the data can be sent to the frontend.

\begin{figure}[h!]
    \centering
      \includegraphics[width=0.9\textwidth]{DataFlowNoBacktracker.png}
      \caption{Dataflow without the Backtracker}}
      \label{dataFlowNoBacktracker}
\end{figure}

\ref{dataFlowNoBacktracker} is a simple application with two transformations much like the WordCount application used throughout this thesis. The first transition is reading the input and the second is modifying it. At each transition, we split the data stream and send an exact copy to another transformation that is part of the Backtracker that sends it to the front end. That means that with the Backtracker enabled the earlier diagram now looks like this: \ref{dataFlowWithBacktracker}

\begin{figure}[h!]
    \centering
      \includegraphics[width=0.9\textwidth]{DataFlowWithBacktracker.png}
      \caption{Dataflow with the Backtracker enabled}
      \label{dataFlowWithBacktracker}
\end{figure}

This ensures that the data is sent to the frontend but fails to show which pieces of information are related to each other. Because of that, it is necessary to add watchpoints to the system. The way that Flink handles watchpoints is simple: They are injected at the beginning of the application as part of the data stream, and each time a transformation receives one it injects the same one back at its output at the same position. Each watchpoint is identified by a unique number and is used as a grouping method in this tool. Flink does not allow the current checkpoint to be read out because there can be multiple active at the same time in different parts of the application. The only way to get the current checkpoint for a given transition is by saving the last checkpoint that was read by the transition. The snapshotState method is called by the Task Manager each time a new watchpoint is registered. The following line then saves the checkpointId as a field of the class so that the most recent watermark is always available:

\begin{lstlisting}[caption={Save Watermark}]
@Override
public void snapshotState(FunctionSnapshotContext context)
throws Exception {
  currentWatermark = context.getCheckpointId();
}
\end{lstlisting}

When sending messages to the frontend, the current watchpoint can then be added.

\paragraph{Sending Data}

Another important step along the way is sending the data to the frontend. This is done by establishing a socket connection to the frontend (the frontend is the server) and sending serialised Java objects over it.

\section{Frontend architecture}
\label{fbFrontend}
The front-end architecture is pretty straightforward. A view is created and populated with a tree view together with a socket server that is awaiting messages from clients. Once it receives a message, it checks if the transformation and/or watchpoint is already present in the tree and fills the data to the correct location. The server can never stop IntelliJ from working as it is running in a separate thread, thus ensuring that any problems with the connection are isolated.


\section{State of the program}
\label{fbState}
The backtracker tool is working, that being said it is not tested in the most vigorous way possible and there are still a few features that would enhance the usability like being able to save the results.

\paragraph{Saving results}
At the moment the only way the data can be displayed is by using the IntelliJ plugin. Because the messages are sent by the internal transformation, a second transformation could be written that instead of sending the data to the IntelliJ plugin saves it in a database or logs it to a file. When creating the Backtracker, a second constructor was needed to specify which Backend function to use.

\paragraph{Better UI}
The UI still lacks a few features that would drastically increase the ease of the program like clearing old results or saving the tree directly from there.


\typeout{===== File: chapter 6}
\chapter{Conclusion and Outlook}
  Summing up all the pieces, this thesis provided a methodology for debugging Flink applications. The basic procedure of a Flink application was discussed, and the pitfalls of writing these programs marked, along the thesis a sample word count application was used to picturise the methods. A tool was written that provides the developer with more information than are available with Flink. Namely which piece of data at which transition lead to which piece of data in the next transition. This allows the developer to gain a better overview over what the program is actually doing compared with what hw thought the program was doing.

\section{Outlook}
Only the data stream part of Flink was discussed in this thesis as Flink also offers a dataset API it would make sense to research how the dataset API might differ in sense of debugging. Although a lot of the same principles still hold true there, it is by no means the same.

\section{Further Reading}
This section will outline a few more articles or books to read that are relevant to the thesis but were not directly mentioned.

\paragraph{} The beginning of the thesis provided some proposals on how to minimise the debugging. It was suggested to use theorem proving or model checking. Because it is not really part of the debugging process it was mostly left out, but because it can be quite helpful for some circumstances it makes sense to read more about it. Amazon uses model checking in some departments and found some bugs that were not known before: \cite{Newcombe:2015:AWS:2749359.2699417}.
A common language for writing these is called TLA+. It is a functional language where the core function of the program is mathematically written. A good introduction can be found here: \cite{Lamport:2002:SST:579617}.


%% appendix if used
%%\appendix
%%\typeout{===== File: appendix}
%%\include{appendix}

% bibliography and other stuff
\backmatter

\typeout{===== Section: literature}
%% read the documentation for customizing the style
\bibliographystyle{dinat}
\bibliography{thesis}

\typeout{===== Section: nomenclature}
%% uncomment if a TOC entry is needed
%%\addcontentsline{toc}{chapter}{Glossar}
\renewcommand{\nomname}{Glossar}
\clearpage
\markboth{\nomname}{\nomname} %% see nomencl doc, page 9, section 4.1
\printnomenclature

%% index
\typeout{===== Section: index}
\printindex

\HAWasurency

\end{document}
