Summing up all the pieces, this thesis provided a methodology for debugging Flink applications. The basic procedure of a Flink application was discussed, and the pitfalls of writing these programs marked, along the thesis a sample word count application was used to picturise the methods. A tool was written that provides the developer with more information than are available with Flink. Namely which piece of data at which transition lead to which piece of data in the next transition. This allows the developer to gain a better overview over what the program is actually doing compared with what hw thought the program was doing.

\section{Outlook}
Only the data stream part of Flink was discussed in this thesis as Flink also offers a dataset API it would make sense to research how the dataset API might differ in sense of debugging. Although a lot of the same principles still hold true there, it is by no means the same.

\section{Further Reading}
This section will outline a few more articles or books to read that are relevant to the thesis but were not directly mentioned.

\paragraph{} The beginning of the thesis provided some proposals on how to minimise the debugging. It was suggested to use theorem proving or model checking. Because it is not really part of the debugging process it was mostly left out, but because it can be quite helpful for some circumstances it makes sense to read more about it. Amazon uses model checking in some departments and found some bugs that were not known before: \cite{Newcombe:2015:AWS:2749359.2699417}.
A common language for writing these is called TLA+. It is a functional language where the core function of the program is mathematically written. A good introduction can be found here: \cite{Lamport:2002:SST:579617}.
