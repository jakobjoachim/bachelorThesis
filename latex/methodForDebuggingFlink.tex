
This chapter is the centre part of the thesis. It explains in detail how the method works. It consists of two main sections, the first of which explores in detail what a developer should do while writing the program to minimise the work when debugging. This step is crucial as it is much harder to recreate problems locally as opposed to regular java applications. The second part focuses on the debugging itself once a developer is informed of an error and has to figure out what is causing it.

\section{Titel to do (Better programming)}
Flink applications are run remotely and without an active user providing input as it is typical for a regular application. Not having a user makes it much harder to recreate the problem as we only have the log files and the stack trace as information. As such it is crucial to have all the information at hand when the program fails, or we notice discrepancies in the resulting data. The only way to make sure that the information is accessible once a problem is reported is to think about what data is necessary for the debugging developer while writing the program. Another issue is that some problems are unique to a distributed environment and won't happen when running on a local host. This section will explain what can be done while writing the program to make the debugging progress much easier.

\subsection{Theorem Proving}


\subsection{Logging and Metrics}
\subsection{Utilising the Flink Tools}
\subsection{Providing backtracking information}

\section{Titel to do (Debugging Flink once an error occurs)}
