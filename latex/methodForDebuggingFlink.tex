
This chapter is the centre part of the thesis. It explains in detail how the method works. It consists of two main sections, the first of which explores in detail what a developer should do while writing the program to minimise the work when debugging. This step is crucial as it is much harder to recreate problems locally as opposed to regular java applications. The second part focuses on the debugging itself once a developer is informed of an error and has to figure out what is causing it.

\section{Better Developing}
Flink applications are run remotely and without an active user providing input as it is typical for a regular application. Not having a user makes it much harder to recreate the problem as we only have the log files and the stack trace as information. As such it is crucial to have all the information at hand when the program fails, or we notice discrepancies in the resulting data. The only way to make sure that the information is accessible once a problem is reported is to think about what data is necessary for the debugging developer while writing the program. Another issue is that some problems are unique to a distributed environment and won't happen when running on a local host. This section will explain what can be done while writing the program to make the debugging progress much easier.

\subsection{Building Tasks}
The Flink Framework promises a few features that try to set it apart from standard Java applications. These are referenced in \ref{flinkFramework}. The one that sets these applications apart from standard java once while programming is the distribution. Applications should be built in a modular way to take full advantage of these features and also make the program easily debuggable.
The development of these tasks should be done using a very similar set of rules as the Unix philosophy states. In fact, Flink modules are not too different to Unix command line tools, they both provide a service while taking an input/output in a predefined way. For the Unix command line, this is the standard output, for Flink programs, these are input and output streams.

\subsubsection{Unix Philosophy}
The Unix philosophy is defined by Doug McIlroy, the inventor of the Unix pipe as follows: \ref{bell1978}
\begin{enumerate}
  \item Make each program do one thing well. To do a new job, build afresh rather than complicate old programs by adding new features.
  \item Expect the output of every program to become the input to another, as yet unknown, program. Don't clutter output with extraneous information. Avoid stringently columnar or binary input formats. Don't insist on interactive input.
  \item Design and build software, even operating systems, to be tried early, ideally within weeks. Don't hesitate to throw away the clumsy parts and rebuild them.
  \item Use tools in preference to unskilled help to lighten a programming task, even if you have to detour to build the tools and expect to throw some of them out after you've finished using them.
\end{enumerate}

Dividing the program into various modules has a lot of advantages:
\begin{enumerate}
  \item Easier to Develop - It is much simpler to develop a smaller application as it is much harder to lose track of what each piece of code should do. This reduced complexity in return reduces the likelihood of mistakes. Once the program is completed, it results in a more stable program that can be debugged easier.
  \item Better Distribution -  As every Flink task can run on a different computer the smaller the tasks are, the better the Flink Job Manager can distribute the load evenly between the available resources.
  \item Checkpoints are easy to find - Checkpoints are a core piece of Flink technology. It allows the framework not only to jump back and repeat a failed run without having to restart the whole application but also provides information about which state the application is currently in. This is extremely helpful as a lot of Flink applications only end when the program is cancelled by the user.
\end{enumerate}

The modulation of the program not only helps to achieve the advantages of the framework but also supports with debugging later as a lot of the information needed are gathered at the checkpoints.

\subsubsection{Where to split the program}
It should now be understandable that the programs should be divided into multiple tasks, the next question now is how to break the program to get a simple program where there are enough tasks but not too many as too many would lead to the opposite effect we want to achieve.

\paragraph{Why are too many tasks bad?} When there are too many tasks, it gets even more complicated than when everything would be in its task as basically every line of code would be in a different place. Additionally, it wouldn't increase the performance as each task has some initialisation work that would diminish the performance gain achieved by distributing it perfectly.

It makes sense to use the Unix philosophy of having one task do one thing. In most cases there are some obvious choices as in most Flink programs data is modified or analysed each task could be one transformation of the data. It should also be stated that when ever possible the pre-existing transformation functions of the Flink framework should be used. Only in rare situations is it necessary to implement your own data transformation classes. The predefined classes have the advantage of being extensively tested in different conditions thus the risk of data going missing is extremely low.

\subsection{Metrics}
Now that the architecture of the program is done the next question is what metrics to use in which positions to achieve the optimal security.

Once the program architecture is finished, it makes sense to think about what kind of metrics can be used where. Metrics are used to monitor the program without having to debug it and are crucial in notifying the developer if something looks wrong. The first step is figuring out where to use metrics. A good start is to look at the application in the worst possible way; what is the most likely area that an error will occur, after that it makes sense to surround the area with metrics that will catch and log the gathered data. Another great location for metrics is at positions where the through coming data is simple, and metrics can easily be implemented. This should be the case in between tasks. As optimally each task only does one thing it should be easy to check whether the starting and ending assertions are valid.

\subsection{Logging}
Logging in Flink is straightforward and can is used the same as in every other java program that uses log4j. As Flink already provides the necessary libraries to use log4j all a developer has to do is to write the logging config file. Although the logging process itself is the same as every other Java application, it should not be forgotten to use the different log levels that log4j provides. There is a lot of logging happening out of the box just by the Flink process itself. The six logging levels, from highest to lowest are:
\begin{enumerate}
  \item FATAL - the highest logging level, should only be used when the application cannot continue to work because of an unexpected error.
  \item ERROR - whenever an unexpected exception is thrown it should be logged.
  \item WARN - warnings that could lead to errors later on. These are difficult to think of beforehand but if used correctly are very valuable for the debugging developer.
  \item INFO - relevant information like successful database connections and other milestones in the application to let the reader of a log file understand at which point in an application the program currently is.
  \item DEBUG - should be used to record relevant information along the way that could be useful to a programmer when debugging. This could, for example, contain values of variables like database connection strings.
  \item TRACE - is used to let a developer searching for a bug understand the path the application took. Should be logged into a unique log file as it would flood every other one.
\end{enumerate}

\section{Titel to do (Debugging Flink once an error occurs)}
