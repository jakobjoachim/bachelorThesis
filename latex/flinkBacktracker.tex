This chapter will explain in detail how the Flink Backtracking tool written along side this thesis works and how it is used. It is composed of two parts, the frontend part that shows the results in the IDE and the backend part that has to be included in the project that sends the information to the frontend. This chapter is thus split into multiple paths:
\begin{enumerate}
  \item[\ref{fbManual}] User manual
  \item[\ref{fbBackend}] Backend architecture
  \item[\ref{fbFrontend}] Frontend architecture
  \item[\ref{fbState}] State of the program
\end{enumerate}

\section{User manual}
\label{fbManual}


\section{Backend architecture}
\label{fbBackend}


\section{Frontend architecture}
\label{fbFrontend}


\section{State of the program}
\label{fbState}
