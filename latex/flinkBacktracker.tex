This chapter will explain in detail how the Flink Backtracking tool written along side this thesis works and how it is used. It is composed of two parts, the frontend part that shows the results in the IDE and the backend part that has to be included in the project that sends the information to the frontend. This chapter is thus split into multiple paths:
\begin{enumerate}
  \item[\ref{fbManual}] User manual
  \item[\ref{fbBackend}] Backend architecture
  \item[\ref{fbFrontend}] Frontend architecture
  \item[\ref{fbState}] State of the program
\end{enumerate}

\section{User manual}
\label{fbManual}

\paragraph{Prerequirements} There are a few prerequirements that have to be met to make the progrgam work appropriatly. First and foremost it has to be a Flink application as it can only track Flink data streams. Also because it is using the watchpoint mechanism of Flink all requiremts that go along with that have to be met. This means each datastream has to support rolling back data in cases of a failure. Because of this requirement it is currently only possible to use data that comes from some sort of big data storage. In the following examples we will use Apache Kafka.

Using the program is an easy process. One has to download the jar and include it in the application. There is only one line needed to enable the tool itself, although checkpointing still has to be enabled manually:

\begin{lstlisting}
env.setStreamTimeCharacteristic(TimeCharacteristic.EventTime);
env.enableCheckpointing(1000);
FlinkBacktrack backtracker = new FlinkBacktrack(env);
\end{lstlisting}

Where "env" is the Execution Environment. Once the Backtracker is initilised it can be used by adding each datastream that the developer wants to be watched to the tool. This is done in the following way:

\begin{lstlisting}
backtracker.track(dataStream);
\end{lstlisting}

Where "dataStream" is the dataStream that should be tracked. This step has to be repeated for each data stream that should be watched.

As well as setting up the application that should be watched one has to also setup intelliJ to receive the data generated by  the plugin. To do that simply activate the attached IntelliJ plugin.

\paragraph{debugging a program with the Backtracker}

Once the plugin has been configured the application can be started. Each piece of data that is being processed by the application will show up in the "Flink Backtracker" tab in IntelliJ as well.

\paragraph{} TODO add picture of example here

The first tabs that are shown are the different data streams. Each watched data stream has its own tab. Inside each of these tabs contains a list of watch points that passed since the program was started and each of these includes the data that was processed by that data stream at the time of the watchpoint.

\section{Backend architecture}
\label{fbBackend}
The

\section{Frontend architecture}
\label{fbFrontend}


\section{State of the program}
\label{fbState}
The backtracker tool is in a working state, that being said it is not tested in the most vigorues way possible and there are still a few features that would really enhance the usability.

\paragraph{Saving results}
At the moment the only way the data can be displayed is by using the IntelliJ plugin. Because of the way the plugin is built it would easily be possible to extend the program to allow multiple saving options.
